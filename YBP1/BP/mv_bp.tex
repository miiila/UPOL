%%%  Ukázkový text a dokumentace stylu pro text závěrečné (bakalářské a
%%%  diplomové) práce na KI PřF UP v Olomouci
%%%  Copyright (C) 2012 Martin Rotter, <rotter.martinos@gmail.com>
%%%  Copyright (C) 2014 Jan Outrata, <jan.outrata@upol.cz>


%%  Pro získání PDF souboru dokumentu je třeba tento zdrojový text v
%%  LaTeXu přeložit (dvakrát) programem pdfLaTeX.

%%  V případě použití programu BibLaTeX pro tvorbu seznamu literatury
%%  je poté ještě třeba spustit program Biber s parametrem jméno
%%  souboru zdrojového textu bez přípony a následně opět (dvakrát)
%%  přeložit zdrojový text programem pdfLaTeX.

%%  Postup získání Postscriptového souboru je popsán v dokumentaci.


%%  Třída dokumentu implementující styl pro závěrečnou práci. Vybrané
%%  nepovinné parametry (ostatní v dokumentaci):

%%  'master' pro sazbu diplomové práce, jinak se sází bakalářská práce

%%  'field=kód' pro Váš studijní obor, kódy pro diplomovou práci 'uvt'
%%  pro Učitelství výpočetní techniky pro střední školy a 'binf' pro
%%  Bioinformatiku, jinak je výchozí Informatika, a pro bakalářskou
%%  práci 'ainfk' pro Aplikovanou informatiku v kombinované formě,
%%  'inf' pro Informatiku, 'infv' pro Informatiku pro vzdělávání a
%%  'binf' pro Bioinfomatiku, jinak je výchozí Aplikovaná informatika
%%  v prezenční formě

%%  'printversion' pro sazbu verze pro tisk (nebarevné logo a odkazy,
%%  odkazy s uvedením adresy za odkazem, ne odkazy do rejstříku),
%%  jinak verze pro prohlížeč

%%  'biblatex' pro zapnutí podpory pro sazbu bibliografie pomocí
%%  BibLaTeXu, jinak je výchozí sazba v prostředí thebibliography

%%  'language=jazyk' pro jazyk práce, jazyky english pro anglický,
%%  slovak pro slovenský, jinak je výchozí czech pro český

%%  'font=sans' pro bezpatkový font (Iwona Light), jinak výchozí
%%  patkový (Latin Modern)

\documentclass[
%  master,
  field=ainfk,
%  printversion,
  biblatex,
%  language=english,
%  font=sans,
  glossaries,
  index
]{kidiplom}

%% Informace pro úvodní strany. V jazyku práce (pokud není v komentáři
%% uvedeno česky) a anglicky. Uveďte všechny, u kterých není v
%% komentáři uvedeno, že jsou volitelné. Při neuvedení se použijí
%% výchozí texty. Text pro jiný než nastavený jazyk práce (nepovinným
%% parametrem language makra \documentclass, výchozí český) se zadává
%% použitím makra s uvedením jazyka jako nepovinného parametru.

\title{Aplikace na tvorbu a správu metadat pro CANopen zařízení}
\title[english]{Metadata management for CANopen devices}

%% Volitelný podnázev práce, česky a anglicky. Měl by se vysázet na
%% jeden řádek. Výchozí je prázdný.
%\subtitle{Ukázkový text a dokumentace stylu v \LaTeX{}u}
%\subtitle[english]{Sample text and documentation of the \LaTeX{} style}

%% Jméno autora práce. Makro nemá nepovinný parametr pro uvedení
%% jazyka.
\author{Miloslav Votradovec}

%% Jméno vedoucího práce (včetně titulů). Makro nemá nepovinný
%% parametr pro uvedení jazyka.
\supervisor{Mgr. Petr Krajča, Ph.D.}
%% Nepovinné - Jméno konzultanta (včetně titulů). Makro nemá nepovinný
%% parametr pro uvedení jazyka.
\consultant{Mgr. Šimon Řeřucha, Ph.D.}

%% Volitelný rok odevzdání práce. Výchozí je aktuální (kalendářní)
%% rok. Makro nemá nepovinný parametr pro uvedení jazyka.
%\yearofsubmit{\the\year}

%% Anotace práce, včetně anglické (obvykle překlad z jazyka
%% práce). Jeden odstavec!
\annotation{Ukázkový text závěrečné práce na Katedře informatiky
  Přírodovědecké fakulty Univerzity Palackého v Olomouci, který je
  zároveň dokumentací stylu pro text práce v \LaTeX{}u. Zdrojový text
  v \LaTeX{}u je doporučeno použít jako šablonu pro text skutečné
  závěrečné práce studenta.}

\annotation[english]{Sample text of thesis at the \kitextdepten,
  \kitextfacultyen, \kitextuniven{} and, at the same time,
  documentation of the \LaTeX{} style for the text. The source text in
  \LaTeX{} is recommended to be used as a template for real student's
  thesis text.}

%% Klíčová slova práce, včetně anglických. Oddělená (obvykle) středníkem.
\keywords{styl textu; závěrečná práce; dokumentace; ukázkový text}
\keywords[english]{text style; thesis; documentation; sample text}

%% Volitelná specifikace příloh textu práce, i anglicky. Výchozí je '1
%% CD/DVD'.
%\supplements{jedno kulaté placaté CD/DVD s malou kulatou dírou uprostřed}
%\supplements[english]{one round flat CD/DVD with a small round hole in the middle}

%% Volitelné poděkování. Stručné! Výchozí je prázdné. Makro nemá
%% nepovinný parametr pro uvedení jazyka.
\thanks{Děkuji, děkuji, děkuji.}

%% Cesta k souboru s bibliografií pro její sazbu pomocí BibLaTeXu
%% (zvolenou nepovinným parametrem biblatex makra
%% \documentclass). Použijte pouze při této sazbě, ne při (výchozí)
%% sazbě v prostředí thebibliography.
\bibliography{bibliografie.bib}

%% Další dodatečné styly (balíky) potřebné pro sazbu vlastního textu
%% práce.
\usepackage{lipsum}

\begin{document}
%% Sazba úvodních stran -- titulní, s bibliografickými údaji, s
%% anotací a klíčovými slovy, s poděkováním a prohlášením, s obsahem a
%% se seznamy obrázků, tabulek, vět a zdrojových kódů (pokud jejich
%% sazba není vypnutá).
\maketitle


%% Vlastní text závěrečné práce. Pro povinné závěry, před přílohami,
%% použijte prostředí kiconclusions. Povinná je i příloha s obsahem
%% přiloženého CD/DVD.

%% -------------------------------------------------------------------

\newcommand{\BibLaTeX}{\textsc{Bib}\LaTeX}

\noindent\textcolor{red}{\LARGE Upozornění: Následující text
  dokumentace stylu, vyjma přílohy~\ref{sec:ObsahCD}, je rozpracovaná
  a (značně) neúplná verze!!!}


\section{Úvod}
\subsection{Cíle práce}
\begin{enumerate}
    \item editor COB/EDS/DCF 
    \item prirucka pro ovady z UPT 
    \item generator kodu 
\end{enumerate}
\section{CAN sběrnice}
\subsection{Historie}
\subsection{Kódování dat}
dominantní - 0, recesivní 1
pokud zařízení odešle 1 a dostane 0, přestává vysílat
zařízení s nižším ID má přednost - 11bit ID je vysíláno na začátku zprávy
ID - 11bit (standard, base) nebo 29bit(extended;11bit base, 18bit rozšíření)
rozlišení podle IDEbitu (14?), dominant pro base, recessive pro extended
\subsection{Struktura rámce}
4 typy rámců
\begin{itemize}
	\item Data
	\item Remote
	\item Error
	\item Overload
\end{itemize}
\subsubsection{Data frame}
%%obrázek rámce (https://en.wikipedia.org/wiki/File:CAN-Bus-frame_in_base_format_without_stuffbits.svg)
rozepsat podle obrázku
\subsubsection{Remote frame}
používá se k vyžádání dat
rozdíl vůči Data frame:
\begin{itemize}
	\item RTR bit (13., Remote transmission request) je 1 (recessive)
	\item DLC označuje velikost vyžádaných dat
	\item neobsahuje Data field
\end{itemize}
Pokud se odešle zároveň Data a Remote se stejným ID, vyhraje Data (kvůli dominantnímu RTR)

\subsubsection{Error frame}
The first field is given by the superposition of ERROR FLAGS (6–12 dominant/recessive bits) contributed from different stations.
The following second field is the ERROR DELIMITER (8 recessive bits).

\subsubsection{Overload frame}

\subsection{Bit Stuffing}
Synchronizace - opačný bit je vložen po pěti stejných bitech. Neplatí pro CRC delimiter, ACK bit a konec rámce. 6 stejných bitů je považováno za chybu
příjemce odstraňuje

\section{Protokol CANopen}

\subsection{Datové typy}

\subsection{Model zařízení}

vztah k ISO/OSI modelu
%%obrázek Figure 3: Communication reference model z 301v04020006_cor3.pdf

Aplikační vrstvě je věnována samostatná podsekce.

\subsubsection{Komunikační vrstva}

Mapování 11bit CAN ID zařízení na 4bit Function ID a 7bit ID zařízení; kritičtější funkce mají nižší ID a tedy vyšší prioritu

Stavový automat, který určuje aktuální stav daného zařízení

\subsubsection{Modely komunikace}

\begin{itemize}
    \item master/slave
    \item client/server
    \item producer/consumer
\end{itemize}

\subsubsection{Slovník objektu}
\begin{itemize}
    \item Index, the 16-bit address of the object in the dictionary
    \item Object name, a symbolic type of the object in the entry, such as an array, record, or simple variable
    \item Name, a string describing the entry
    \item Type, gives the datatype of the variable (or the datatype of all variables of an array)
    \item Attribute, which gives information on the access rights for this entry, this can be read/write, read-only or write-only
    \item The Mandatory/Optional field (M/O) defines whether a device conforming to the device specification has to implement this object or not 
\end{itemize}

\subsection{Aplikační vrstva}

\subsubsection{Mapa adresního prostoru}
%% www.microcontrol.net/download/appnotes/td-03011e.pdf, pp3

\subsubsection{NMT}
Mění stavy automatu a tedy stavy celého zařízení

\begin{itemize}
    \item 0x01  Go to 'operational'
    \item 0x02  Go to 'stopped'
    \item 0x80  Go to 'pre-operational'
    \item 0x81  Go to 'reset node'
    \item 0x82  Go to 'reset communication'
\end{itemize}

Součástí je Heartbeat - monitoruje zařízení v sítí
dve metody HB, periodic signaling + request/response (v CANopen terminologii se to asi jmenuje jinak) - ruzne kody pro hlaseni stavu a prikazy pro zmenu stavu

\subsubsection{SDO}
Používá se ke čtení a nastavování hodnot v Object dictionary. Přistupuje klient k serveru (komunikaci tedy zahajuje klient). Segmentation + desegmentation. COB ID Např. Configuration settings, possibly node ID, baud rate, offset, gain, etc.

\subsubsection{PDO}
Real time data v různých uzlech. 8bytů dat. PDO může obsahovat více vstupů z Object Dictionary. TPDO (transmit), RPDO (receive), většinou data ze senzorů jako Values of type rotational speed, voltage, frequency, electric current apod.

\subsubsection{SYNC}
Synchronizační signál

\subsubsection{TIME}
Usually the Time-Stamp object represents an absolute time in milliseconds after midnight and the number of days since January 1, 1984. This is a bit sequence of length 48 (6 bytes).

\subsubsection{EMCY}
Emergency object - pokud nastane Fatal error

\subsection{Electronic Data Sheet}
EDS - formát souboru, který popisuje komunikaci a záznamy v Object Dictionary v daném zařízení. Umožňuje přístup k ovládání a nastavovení daného zařízení. Jsou vyžadovány pro každé zařízení, které má spadat pod CANOPEN standard.

\section{Manufacturer-specific proprietary extensions}

\subsection{canopen}

\begin{itemize}
    \item id schema
    \item pdo bez PDO mappingu, misto toho mame COB
    \item sdo bez "dlouhych transferu"
    \item nmt ve verzi H
    \item pouze podmnozina datovych typu
    \item EDS pro definici object dict
\end{itemize}

\subsection{SW tooly}

\begin{itemize}
    \item netcans - server, packet reflector: canopen <-> tcp/ip, umi logovat provoz 
    \item canusb32.dll - parsuje metadata site, C-based API mezi SW tooly a netcans
    \item canman - manazer, komunikuje s netcans skrz canusb32.dll, zobrazuje PDO, umoznuje SDO operace
    \item uCan - fyzicky prevodnik fyz.can <-> netcans
    \item etherCan - dtto, po eth
    \item kanal - parsuje log z netcans, parsuje metadata site, exportuje zaznamy hodnot 
    \item canproxy - bridge mezi dvema fyzickymi CAN segmenty, na urovni zarizeni, jednosmerne
    \item virtualni instrumenty - chovaji se jako CAN zarizeni, ale nejsou pripojeny na fyzicky can, pouze na netcans
\end{itemize}

\subsection{metody}

\begin{itemize}
    \item k EDS mame DCF, obraz hodnot konkretniho zarizeni 
    \item pouzivame standardizovane "sady" zaznamu v obj dict
    \item COB soubory, supluje/nahrazuje PDO mapping v CanOPen
    \item CAN soubory, definice zarizeni na jednom segmentu, vyuziva jej canusb32.dll a skrz ni i canman, canproxy.... 
    \item cislovani promennych
\end{itemize}

\section{Specifikace programu}

\section{Analýza a návrh programu}

\section{Implementace}

%% Závěry práce. V jazyce práce a anglicky. Text pro jiný než
%% nastavený jazyk práce (nepovinným parametrem language makra
%% \documentclass, výchozí český) se zadává použitím makra s uvedením
%% jazyka jako nepovinného parametru.
\begin{kiconclusions}
Závěr práce v \uv{českém} jazyce.
\end{kiconclusions}

\begin{kiconclusions}[english]
Thesis conclusions in \uv{English}.
\end{kiconclusions}

%% Přílohy obsahu textu práce, za makrem \appendix.
\appendix

\section{První příloha}
Text první přílohy

\section{Druhá příloha}
Text druhé přílohy

%% -------------------------------------------------------------------

%% Sazba volitelného seznamu zkratek, za přílohami.
\printglossary

%% Sazba povinné bibliografie, za přílohami (případně i za seznamem
%% zkratek). Při použití BibLaTeXu použijte makro
%% \printbibliography. jinak prostředí thebibliography. Ne obojí!

%% Sazba i v textu necitovaných zdrojů, při použití
%% BibLaTeXu. Volitelné.
\nocite{*}
%% Vlastní sazba bibliografie při použití BibLaTeXu.
%%\printbibliography

%% Bibliografie, včetně sazby, při nepoužití BibLaTeXu.
% \begin{thebibliography}{9}
%\bibitem{kniha2} \uppercase{Hawke}, Paul. NanoHttpd: Light-weight HTTP server designed for embedding in other applications. GitHub [online]. 2014-05-12. [cit. 2014-12-06]. Dostupné z: \url{https://github.com/NanoHttpd/nanohttpd}
%
%\bibitem{jeske13} \uppercase{Jeske}, David; \uppercase{Novák}, Josef. Simple HTTP Server in \csharp: Threaded synchronous HTTP Server abstract class, to respond to HTTP requests. CodeProject: For those who code [online]. 2014-05-24. [cit. 2014-12-06]. Dostupné z: \url{http://www.codeproject.com/Articles/137979/Simple-HTTP-Server-in-C}
%
%\bibitem{uzis2012} \uppercase{ÚSTAV ZDRAVOTNICKÝCH INFORMACÍ A STATISTIKY ČR}. Lékaři, zubní lékaři a farmaceuti 2012 [online]. Praha 2, Palackého náměstí 4: Ústav zdravotnických informací a statistiky ČR, 2012 [cit. 2014-12-06]. ISBN 978-80-7472-089-5. Dostupné z: \url{http://www.uzis.cz/publikace/lekari-zubni-lekari-farmaceuti-2012}
% \end{thebibliography}

%% Sazba volitelného rejstříku, za bibliografií.
\printindex

\end{document}
